%% Generated by Sphinx.
\def\sphinxdocclass{report}
\documentclass[a4paper,10pt,english]{sphinxmanual}
\ifdefined\pdfpxdimen
   \let\sphinxpxdimen\pdfpxdimen\else\newdimen\sphinxpxdimen
\fi \sphinxpxdimen=.75bp\relax
\ifdefined\pdfimageresolution
    \pdfimageresolution= \numexpr \dimexpr1in\relax/\sphinxpxdimen\relax
\fi
%% let collapsable pdf bookmarks panel have high depth per default
\PassOptionsToPackage{bookmarksdepth=5}{hyperref}
%% turn off hyperref patch of \index as sphinx.xdy xindy module takes care of
%% suitable \hyperpage mark-up, working around hyperref-xindy incompatibility
\PassOptionsToPackage{hyperindex=false}{hyperref}
%% memoir class requires extra handling
\makeatletter\@ifclassloaded{memoir}
{\ifdefined\memhyperindexfalse\memhyperindexfalse\fi}{}\makeatother

\PassOptionsToPackage{warn}{textcomp}

\catcode`^^^^00a0\active\protected\def^^^^00a0{\leavevmode\nobreak\ }
\usepackage{cmap}
\usepackage{xeCJK}
\usepackage{amsmath,amssymb,amstext}
\usepackage{babel}



\setmainfont{FreeSerif}[
  Extension      = .otf,
  UprightFont    = *,
  ItalicFont     = *Italic,
  BoldFont       = *Bold,
  BoldItalicFont = *BoldItalic
]
\setsansfont{FreeSans}[
  Extension      = .otf,
  UprightFont    = *,
  ItalicFont     = *Oblique,
  BoldFont       = *Bold,
  BoldItalicFont = *BoldOblique,
]
\setmonofont{FreeMono}[
  Extension      = .otf,
  UprightFont    = *,
  ItalicFont     = *Oblique,
  BoldFont       = *Bold,
  BoldItalicFont = *BoldOblique,
]



\usepackage[Sonny]{fncychap}
\ChNameVar{\Large\normalfont\sffamily}
\ChTitleVar{\Large\normalfont\sffamily}
\usepackage{sphinx}

\fvset{fontsize=\small,formatcom=\xeCJKVerbAddon}
\usepackage{geometry}


% Include hyperref last.
\usepackage{hyperref}
% Fix anchor placement for figures with captions.
\usepackage{hypcap}% it must be loaded after hyperref.
% Set up styles of URL: it should be placed after hyperref.
\urlstyle{same}

\addto\captionsenglish{\renewcommand{\contentsname}{Contents:}}

\usepackage{sphinxmessages}
\setcounter{tocdepth}{1}


\usepackage{ctex}


\title{无线PLC}
\date{2022 年 03 月 24 日}
\release{V1.0.2}
\author{LHC@云南兆富科技有限公司}
\newcommand{\sphinxlogo}{\vbox{}}
\renewcommand{\releasename}{发布}
\makeindex
\begin{document}

\ifdefined\shorthandoff
  \ifnum\catcode`\=\string=\active\shorthandoff{=}\fi
  \ifnum\catcode`\"=\active\shorthandoff{"}\fi
\fi

\pagestyle{empty}
\sphinxmaketitle
\pagestyle{plain}
\sphinxtableofcontents
\pagestyle{normal}
\phantomsection\label{\detokenize{index::doc}}


\begin{figure}[htbp]
\centering

\noindent\sphinxincludegraphics[scale=0.4]{{01_开机界面}.jpg}
\end{figure}
\begin{itemize}
\item {} 
\sphinxAtStartPar
\sphinxstylestrong{公司主页(homepage)}:\sphinxurl{http://www.ynpax.com/cn/home/index.asp}

\item {} 
\sphinxAtStartPar
\sphinxstylestrong{合作方主页}:\sphinxurl{http://www.yn.csg.cn/}

\item {} 
\sphinxAtStartPar
\sphinxstylestrong{公司地址}:云南省昆明市经开区云大西路39号创业大厦C栋202室

\item {} 
\sphinxAtStartPar
\sphinxstylestrong{联系电话}:0871\sphinxhyphen{}6732300/67322190

\end{itemize}


\chapter{产品概述}
\label{\detokenize{Product_Overview:id1}}\label{\detokenize{Product_Overview::doc}}
\sphinxAtStartPar
WLC\sphinxhyphen{}1210系列是一种可编程序逻辑控制器(Micro PLC)。它能够控制各种设备以满足自动化控制需求。
WLC\sphinxhyphen{}1210的用户程序中包括了位逻辑、计数器、定时器、复杂数学运算以及与其它智能模块通讯等指令内容,从而使它能够监视输入状态,改变输出状态以达到控制目的。紧凑的结构、灵活的配置和强 大的指令集使 \sphinxstylestrong{WLC\sphinxhyphen{}1210} 成为各种控制应用的理想解决方案。

\begin{figure}[htbp]
\centering
\capstart

\noindent\sphinxincludegraphics[scale=0.8]{{指示图}.jpg}
\caption{图2.1.1  WLC\sphinxhyphen{}1210系列PLC的外部结构}\label{\detokenize{Product_Overview:id4}}\end{figure}


\section{WLC\sphinxhyphen{}1210的外部结构}
\label{\detokenize{Product_Overview:wlc-1210}}
\sphinxAtStartPar
WLC\sphinxhyphen{}1210将微处理器、集成电源、输入电路和输出电路集为一体,从而形成了一个功能强大的Micro PLC。参见图2.1.1。在下载了程序之后,WLC\sphinxhyphen{}1210将保留所需的逻辑,用于监控应用程序中的输入输出设备。

\sphinxAtStartPar
\sphinxstylestrong{1、输入端子}

\sphinxAtStartPar
输入端子是外部输入信号与PLC连接的接线端子,位于底部端盖下面。此外,外部端盖下面还有输入公共端子。输入端子工作电压为直流24V。

\begin{figure}[htbp]
\centering
\capstart

\noindent\sphinxincludegraphics[scale=0.1]{{输入端子}.jpg}
\caption{图2.1.2  WLC\sphinxhyphen{}1210系列 PLC的输入端子}\label{\detokenize{Product_Overview:id5}}\end{figure}

\sphinxAtStartPar
\sphinxstylestrong{2、输出端子}

\sphinxAtStartPar
输出端子是外部负载与PLC连接的接线端子,位于顶部端盖下面。此外,顶部端盖下面还有输出公共端子和PLC工作电源端子,电源为直流24V。

\begin{figure}[htbp]
\centering
\capstart

\noindent\sphinxincludegraphics[scale=0.1]{{输出端子}.jpg}
\caption{图2.1.3  WLC\sphinxhyphen{}1210系列PLC的输出端子}\label{\detokenize{Product_Overview:id6}}\end{figure}

\sphinxAtStartPar
\sphinxstylestrong{3、输入状态指示灯(LED)}

\sphinxAtStartPar
输入状态指示灯用于显示是否有输入控制信号接入PLC。当指示灯亮时,表示有控制信号接入PLC;当指示灯不亮时,表示没有控制信号接入PLC。

\begin{figure}[htbp]
\centering
\capstart

\noindent\sphinxincludegraphics[scale=0.3]{{输入状态指示灯}.jpg}
\caption{图2.1.4  WLC\sphinxhyphen{}1210系列PLC的输入状态指示灯}\label{\detokenize{Product_Overview:id7}}\end{figure}

\sphinxAtStartPar
\sphinxstylestrong{4、输出状态指示灯(LED)}

\sphinxAtStartPar
输出状态指示灯用于显示是否有输出信号驱动执行设备。当指示灯亮时,表示有输出信号驱动外部设备;当指示灯不亮时,表示没有输出信号驱动外部设备。

\begin{figure}[htbp]
\centering
\capstart

\noindent\sphinxincludegraphics[scale=0.3]{{输出状态指示灯}.jpg}
\caption{图2.1.5  WLC\sphinxhyphen{}1210系列PLC的输出状态指示灯}\label{\detokenize{Product_Overview:id8}}\end{figure}

\sphinxAtStartPar
\sphinxstylestrong{5、CPU状态指示灯(LED)}

\sphinxAtStartPar
CPU状态指示灯有RUN、STOP、ERROR三个,其中RUN、STOP指示灯用于显示当前工作状态。当RUN指示灯亮时,表示运行状态;当STOP指示灯亮时,表示停止状态;当ERROR指示灯亮时,表示系统故障,PLC停止工作。

\begin{figure}[htbp]
\centering
\capstart

\noindent\sphinxincludegraphics[scale=0.3]{{CPU状态指示灯}.jpg}
\caption{图2.1.6  WLC\sphinxhyphen{}1210系列PLC的CPU状态指示灯}\label{\detokenize{Product_Overview:id9}}\end{figure}

\sphinxAtStartPar
\sphinxstylestrong{6、扩展接口}

\sphinxAtStartPar
扩展接口包括485通信接口(485A,485B)、电压输出端口AVo(AVo:0\sphinxhyphen{}10V)、电流输出端口AIo(4\sphinxhyphen{}20mA)、两个输入VIN0、VIN2。

\begin{figure}[htbp]
\centering
\capstart

\noindent\sphinxincludegraphics[scale=0.15]{{扩展接口}.jpg}
\caption{图2.1.7  WLC\sphinxhyphen{}1210系列PLC的扩展接口}\label{\detokenize{Product_Overview:id10}}\end{figure}

\sphinxAtStartPar
\sphinxstylestrong{7、通信接口}

\sphinxAtStartPar
通信接口PORT0支持PPI、MODBUS通信协议,有自由方式通信能力,通过通信电缆实现PLC与编程器之间、PLC与计算机之间、PLC与PLC之间、PLC与其他设备之间的通信。

\begin{figure}[htbp]
\centering
\capstart

\noindent\sphinxincludegraphics[scale=0.3]{{通信接口}.jpg}
\caption{图2.1.8  WLC\sphinxhyphen{}1210系列PLC的通信接口}\label{\detokenize{Product_Overview:id11}}\end{figure}


\section{WLC\sphinxhyphen{}1210的安装}
\label{\detokenize{Product_Overview:id2}}
\sphinxAtStartPar
可以在一个面板或标准DIN导轨上安装WLC\sphinxhyphen{}1210,WLC\sphinxhyphen{}1210可采用水平或垂直方式安装。

\begin{sphinxadmonition}{note}{注解:}\begin{itemize}
\item {} 
\sphinxAtStartPar
WLC\sphinxhyphen{}1210 PLC是开放式控制器。它要求在外壳、机柜或电气控制室中安装WLC\sphinxhyphen{}1210。只有授权人员才能进入壳、机柜或电气控制室。

\item {} 
\sphinxAtStartPar
不遵守这些安装要求会导致人员死亡或重伤,和/或损坏设备。

\item {} 
\sphinxAtStartPar
当安装WLC\sphinxhyphen{}1210 PLC时始终遵守这些要求。

\end{itemize}
\end{sphinxadmonition}

\sphinxAtStartPar
\sphinxstylestrong{1、将WLC\sphinxhyphen{}1210与热源、高电压和电子噪声隔开}
\begin{itemize}
\item {} 
\sphinxAtStartPar
按照惯例,在安装元器件时,总是把产生高电压和高电子噪声设备与诸如WLC\sphinxhyphen{}1210这样的低压、逻辑型的设备分隔开。

\item {} 
\sphinxAtStartPar
在控制柜背板上安排WLC\sphinxhyphen{}1210时,应区分发热装置并把电子器件安排在控制柜中温度较低的区域内。电子器件在高温环境下工作会缩短其无故障时间。

\item {} 
\sphinxAtStartPar
还要考虑面板中设备的布线。避免将低压信号和通讯电缆与交流供电线和高能量、开关频率很高的直流线路布置在一个线槽中。

\end{itemize}

\sphinxAtStartPar
\sphinxstylestrong{2、先决条件}
\begin{itemize}
\item {} 
\sphinxAtStartPar
在安装和拆卸任何电气设备之前,必须确认该设备的电源已断开。同样,也要确保与该设备关联的设备供电已被切断。

\item {} 
\sphinxAtStartPar
试图在带电情况下安装或拆卸WLC\sphinxhyphen{}1210及其相关设备有可能导致触电或者设备误动作。

\end{itemize}

\sphinxAtStartPar
\sphinxstylestrong{3、WLC\sphinxhyphen{}1210的外部结构与接线}

\begin{figure}[htbp]
\centering
\capstart

\noindent\sphinxincludegraphics[scale=0.6]{{外部结构与接线图}.jpg}
\caption{图2.2.1  WLC\sphinxhyphen{}1210系列PLC的外部结构与接线图}\label{\detokenize{Product_Overview:id12}}\end{figure}

\sphinxAtStartPar
图2.2.1是WLC\sphinxhyphen{}1210的外部接线图。在PLC编程中,外部接线图也是其重要的组成部分。
\begin{itemize}
\item {} 
\sphinxAtStartPar
① 输入端子:WLC\sphinxhyphen{}1210共有12点输入,端子编号采用8进制。输入端子只有一组,I0.0\sphinxhyphen{}I1.3,其公共端为COM。输入端工作电压为直流24V。

\end{itemize}

\begin{figure}[htbp]
\centering
\capstart

\noindent\sphinxincludegraphics[scale=0.5]{{输入接线图}.jpg}
\caption{图2.2.2  WLC\sphinxhyphen{}1210系列PLC的输入接线图}\label{\detokenize{Product_Overview:id13}}\end{figure}
\begin{itemize}
\item {} 
\sphinxAtStartPar
② 输出端子:WLC\sphinxhyphen{}1210共有10点输出,端子编号也采用8进制。输出端子共分为3组,Q0.0\sphinxhyphen{}Q0.2为第一组,公共端CM1;Q0.3\sphinxhyphen{}Q0.6为第二组,公共端CM2;Q0.7\sphinxhyphen{}Q1.1为第三组,公共端CM3。根据负载性质不同,其输出回路电源支持交流和直流。

\end{itemize}

\begin{figure}[htbp]
\centering
\capstart

\noindent\sphinxincludegraphics[scale=0.5]{{输出接线图}.jpg}
\caption{图2.2.3  WLC\sphinxhyphen{}1210系列PLC的输出接线图}\label{\detokenize{Product_Overview:id14}}\end{figure}
\begin{itemize}
\item {} 
\sphinxAtStartPar
③ 扩展端口:扩展接口包括485通信接口(485A,485B)、电压输出端口AVo(AVo:0\sphinxhyphen{}10V)、电流输出端口AIo(4\sphinxhyphen{}20mA)、两个输入VIN0、VIN2。

\end{itemize}

\begin{figure}[htbp]
\centering
\capstart

\noindent\sphinxincludegraphics[scale=0.45]{{扩展端口接线图}.jpg}
\caption{图2.2.4  WLC\sphinxhyphen{}1210系列PLC的扩展端口接线图}\label{\detokenize{Product_Overview:id15}}\end{figure}
\begin{itemize}
\item {} 
\sphinxAtStartPar
④ 电源接线:24V直流电源接入。

\end{itemize}

\begin{figure}[htbp]
\centering
\capstart

\noindent\sphinxincludegraphics[scale=0.5]{{电源接线图}.jpg}
\caption{图2.2.5  WLC\sphinxhyphen{}1210系列PLC的电源接线图}\label{\detokenize{Product_Overview:id16}}\end{figure}


\section{WLC\sphinxhyphen{}1210的软件使用说明}
\label{\detokenize{Product_Overview:id3}}
\sphinxAtStartPar
WLC\sphinxhyphen{}1210是全兼容西门子S7\sphinxhyphen{}200系列CPU224XP的物联网PLC,使用STEP7–Micro/WIN编程软件。PLC类型选择CPU 224XP,如下图所示。

\begin{figure}[htbp]
\centering
\capstart

\noindent\sphinxincludegraphics[scale=0.55]{{STEP7选择PLC类型}.jpg}
\caption{图2.2.6  STEP7–Micro/WIN 选择PLC类型}\label{\detokenize{Product_Overview:id17}}\end{figure}

\sphinxAtStartPar
其余详细内容,请参考西门子S7\sphinxhyphen{}200编程手册。


\chapter{操作指南}
\label{\detokenize{operation_guide:id1}}\label{\detokenize{operation_guide::doc}}

\section{上电开机}
\label{\detokenize{operation_guide:id2}}
\begin{figure}[htbp]
\centering
\capstart

\noindent\sphinxincludegraphics[scale=0.5]{{01_开机界面}.jpg}
\caption{图 1.1.1 开机界面}\label{\detokenize{operation_guide:id3}}\end{figure}

\sphinxAtStartPar
在右上角接入 \sphinxstylestrong{24V} 直流电源后,\sphinxcode{\sphinxupquote{POWER}} 灯将会亮起,表示供电正常,您将会看到以上界面。


\section{PLC运行参数配置}
\label{\detokenize{operation_guide:plc}}\begin{itemize}
\item {} 
\sphinxAtStartPar
在对PLC进行编程之前,我们需要对PLC运行时一些必要的系统参数进行设置,设置步骤如下:

\end{itemize}

\sphinxAtStartPar
\sphinxstylestrong{1、进入菜单栏}
\begin{itemize}
\item {} 
\sphinxAtStartPar
① 首先需要在开机界面按下 \sphinxcode{\sphinxupquote{MENU}} 按钮进入密码输入界面。此时您需要判断是否设置过密码,如果设置过密码,则通过 \sphinxcode{\sphinxupquote{DOWN}} (数字0\sphinxhyphen{}9递增) 和 \sphinxcode{\sphinxupquote{UP}} (数字0\sphinxhyphen{}9递减)按钮输入4位密码。

\item {} 
\sphinxAtStartPar
② 值得注意的是,每输入一位密码我们都需要按下 \sphinxcode{\sphinxupquote{MENU}} 按钮进行单位确认操作,否则操作可能无效,最后所有密码输入完毕后在按下 \sphinxcode{\sphinxupquote{ENTER}} ,密码无误将会进入菜单界面。

\item {} 
\sphinxAtStartPar
③ 另外一种情况是用户如果是首次使用该设备,则不需要输入任何密码,直接在密码输入界面按下 \sphinxcode{\sphinxupquote{ENTER}} 即可。

\item {} 
\sphinxAtStartPar
④ 如果您前期设置过密码,后面觉得不需要了,可以通过 菜单栏中的 \sphinxcode{\sphinxupquote{恢复出厂设置}} 项解除密码。

\end{itemize}

\begin{figure}[htbp]
\centering
\capstart

\noindent\sphinxincludegraphics[scale=0.5]{{13_密码输入界面}.jpg}
\caption{图 1.2.1 密码输入界面}\label{\detokenize{operation_guide:id4}}\end{figure}

\begin{figure}[htbp]
\centering
\capstart

\noindent\sphinxincludegraphics[scale=0.5]{{02_主菜单1}.jpg}
\caption{图 1.2.2 菜单界面1}\label{\detokenize{operation_guide:id5}}\end{figure}

\sphinxAtStartPar
\sphinxstylestrong{2、设置PLC运行状态}

\begin{figure}[htbp]
\centering
\capstart

\noindent\sphinxincludegraphics[scale=0.5]{{06_PLC启动}.jpg}
\caption{图 1.3.1 PLC启动}\label{\detokenize{operation_guide:id6}}\end{figure}
\begin{itemize}
\item {} 
\sphinxAtStartPar
在 \sphinxstylestrong{图 1.2.2 菜单界面1} 中选中 \sphinxcode{\sphinxupquote{启停开关}} 后连续按下 \sphinxstylestrong{2} 次 \sphinxcode{\sphinxupquote{ENTER}} 就会看到 \sphinxstylestrong{图 1.3.1 PLC启动} 界面,
此时如果 \sphinxcode{\sphinxupquote{RUN}} 灯亮起,则说明PLC启动成功,否则PLC \sphinxcode{\sphinxupquote{故障}}。

\end{itemize}

\begin{sphinxadmonition}{note}{注解:}\begin{itemize}
\item {} 
\sphinxAtStartPar
PLC未能成功启动的原因可能如下:

\item {} 
\sphinxAtStartPar
\sphinxcode{\sphinxupquote{RUN}} 指示灯故障,导致启动成功,但是状态指示灯没有亮起。

\item {} 
\sphinxAtStartPar
PLC程序块损坏,导致PLC无法切换至运行状态。

\item {} 
\sphinxAtStartPar
PLC的硬件启停线路损坏,无法正常启停。

\item {} 
\sphinxAtStartPar
PLC系统受不可抗拒因素造成损坏。

\end{itemize}
\end{sphinxadmonition}
\begin{itemize}
\item {} 
\sphinxAtStartPar
在 \sphinxcode{\sphinxupquote{图 1.3.1 PLC启动}} 中再次按下 \sphinxcode{\sphinxupquote{ENTER}} 则PLC处于停止状态。

\end{itemize}

\begin{figure}[htbp]
\centering
\capstart

\noindent\sphinxincludegraphics[scale=0.5]{{05_PLC停止}.jpg}
\caption{图 1.3.2 PLC停止}\label{\detokenize{operation_guide:id7}}\end{figure}

\begin{sphinxadmonition}{note}{注解:}\begin{itemize}
\item {} 
\sphinxAtStartPar
PLC处于停止状态时,PLC的程序块可能不会正常工作!

\item {} 
\sphinxAtStartPar
您在没有对PLC的 \sphinxcode{\sphinxupquote{Port0}} 和 \sphinxcode{\sphinxupquote{Port1}} 做其他通信初始化时,并且菜单项 \sphinxcode{\sphinxupquote{通信协议}} 处于 \sphinxcode{\sphinxupquote{PPI\_P}} 时
默认的 \sphinxcode{\sphinxupquote{PPI}} 协议将不会受到影响,处于启用状态。

\end{itemize}
\end{sphinxadmonition}

\sphinxAtStartPar
\sphinxstylestrong{3、设置通信方式}

\sphinxAtStartPar
在 \sphinxstylestrong{图 1.2.2 菜单界面1} 中选中 \sphinxcode{\sphinxupquote{通信方式}} 后按下 \sphinxcode{\sphinxupquote{ENTER}} 将会进入如下界面:

\begin{figure}[htbp]
\centering
\capstart

\noindent\sphinxincludegraphics[scale=0.5]{{07_通信方式_打开}.jpg}
\caption{图 1.4.1 打开无线模块}\label{\detokenize{operation_guide:id8}}\end{figure}
\begin{itemize}
\item {} 
\sphinxAtStartPar
\sphinxstylestrong{无线模块} 即我们内部的WIFI模块,默认该硬件模块处于启用状态。

\end{itemize}

\begin{sphinxadmonition}{note}{注解:}\begin{itemize}
\item {} 
\sphinxAtStartPar
WIFI模块处于启用状态并且接入指定路由设备,连接至云端后,可以看到 \sphinxcode{\sphinxupquote{NET}} 和 \sphinxcode{\sphinxupquote{WAN}} 指示灯将会亮起。

\end{itemize}
\end{sphinxadmonition}

\sphinxAtStartPar
在 \sphinxstylestrong{图 1.4.1 打开无线模块} 选择 \sphinxcode{\sphinxupquote{关闭无线模块}}, 则会得到如下界面:

\begin{figure}[htbp]
\centering
\capstart

\noindent\sphinxincludegraphics[scale=0.5]{{08_通信方式_关闭}.jpg}
\caption{图 1.4.2 关闭无线模块}\label{\detokenize{operation_guide:id9}}\end{figure}

\begin{sphinxadmonition}{note}{注解:}\begin{itemize}
\item {} 
\sphinxAtStartPar
关闭无线模块后,WIIF模块处于禁用状态,PLC的功耗将会降低,可以看到 \sphinxcode{\sphinxupquote{NET}} 和 \sphinxcode{\sphinxupquote{WAN}} 指示灯熄灭。

\end{itemize}
\end{sphinxadmonition}

\sphinxAtStartPar
\sphinxstylestrong{4、设置工作模式}

\sphinxAtStartPar
工作模式主要针对的是在使用 \sphinxstylestrong{外部扩展口} 进行通讯时,实际的PLC硬件口 \sphinxstylestrong{PORT0} (内部网络)的工作模式。


\begin{savenotes}\sphinxattablestart
\centering
\sphinxcapstartof{table}
\sphinxthecaptionisattop
\sphinxcaption{表1.1.1 PLC作为SAVLE}\label{\detokenize{operation_guide:id10}}
\sphinxaftertopcaption
\begin{tabular}[t]{|\X{10}{90}|\X{20}{90}|\X{20}{90}|\X{20}{90}|\X{20}{90}|}
\hline
\sphinxstyletheadfamily 
\sphinxAtStartPar
请求对象
&\sphinxstyletheadfamily 
\sphinxAtStartPar
工作模式
&\sphinxstyletheadfamily 
\sphinxAtStartPar
响应对象
&\sphinxstyletheadfamily 
\sphinxAtStartPar
默认波特率
&\sphinxstyletheadfamily 
\sphinxAtStartPar
代号
\\
\hline
\sphinxAtStartPar
WIFI
&
\sphinxAtStartPar
分时复用
&
\sphinxAtStartPar
PLC\_Port0
&
\sphinxAtStartPar
115200
&
\sphinxAtStartPar
无线网
\\
\hline
\sphinxAtStartPar
LAN
&
\sphinxAtStartPar
分时复用
&
\sphinxAtStartPar
PLC\_Port0
&
\sphinxAtStartPar
115200
&
\sphinxAtStartPar
以太网
\\
\hline
\sphinxAtStartPar
RS485
&
\sphinxAtStartPar
分时复用
&
\sphinxAtStartPar
PLC\_Port0
&
\sphinxAtStartPar
9600
&
\sphinxAtStartPar
扩展网
\\
\hline
\end{tabular}
\par
\sphinxattableend\end{savenotes}

\begin{figure}[htbp]
\centering
\capstart

\noindent\sphinxincludegraphics[scale=0.5]{{09_工作模式_从机}.jpg}
\caption{图 1.5.1 PLC作Savle}\label{\detokenize{operation_guide:id11}}\end{figure}


\begin{savenotes}\sphinxattablestart
\centering
\sphinxcapstartof{table}
\sphinxthecaptionisattop
\sphinxcaption{表1.1.2 PLC作为MASTER}\label{\detokenize{operation_guide:id12}}
\sphinxaftertopcaption
\begin{tabular}[t]{|\X{10}{90}|\X{20}{90}|\X{20}{90}|\X{20}{90}|\X{20}{90}|}
\hline
\sphinxstyletheadfamily 
\sphinxAtStartPar
请求对象
&\sphinxstyletheadfamily 
\sphinxAtStartPar
工作模式
&\sphinxstyletheadfamily 
\sphinxAtStartPar
响应对象
&\sphinxstyletheadfamily 
\sphinxAtStartPar
默认波特率
&\sphinxstyletheadfamily 
\sphinxAtStartPar
代号
\\
\hline
\sphinxAtStartPar
PLC\_Port0
&
\sphinxAtStartPar
分时复用
&
\sphinxAtStartPar
WIFI
&
\sphinxAtStartPar
115200
&
\sphinxAtStartPar
内部网
\\
\hline
\sphinxAtStartPar
PLC\_Port0
&
\sphinxAtStartPar
分时复用
&
\sphinxAtStartPar
LAN
&
\sphinxAtStartPar
115200
&
\sphinxAtStartPar
内部网
\\
\hline
\sphinxAtStartPar
PLC\_Port0
&
\sphinxAtStartPar
分时复用
&
\sphinxAtStartPar
RS485
&
\sphinxAtStartPar
9600
&
\sphinxAtStartPar
扩展网
\\
\hline
\end{tabular}
\par
\sphinxattableend\end{savenotes}

\begin{figure}[htbp]
\centering
\capstart

\noindent\sphinxincludegraphics[scale=0.5]{{10_工作模式_主机}.jpg}
\caption{图 1.5.2 PLC作Mster}\label{\detokenize{operation_guide:id13}}\end{figure}

\begin{sphinxadmonition}{note}{注解:}\begin{itemize}
\item {} 
\sphinxAtStartPar
对于PLC作为 \sphinxstylestrong{Master} 时,目前对应关系仅开放了 \sphinxstylestrong{表1.1.2} 中第三钟模式。

\end{itemize}
\end{sphinxadmonition}

\sphinxAtStartPar
\sphinxstylestrong{5、设置通讯协议}

\sphinxAtStartPar
通讯协议主要针对的是在使用 \sphinxstylestrong{外部扩展口} 进行通讯时,扩展口默认走的是PLC的 \sphinxstylestrong{PPI协议}。


\begin{savenotes}\sphinxattablestart
\centering
\sphinxcapstartof{table}
\sphinxthecaptionisattop
\sphinxcaption{表1.1.3 协议说明}\label{\detokenize{operation_guide:id14}}
\sphinxaftertopcaption
\begin{tabular}[t]{|\X{10}{60}|\X{30}{60}|\X{20}{60}|}
\hline
\sphinxstyletheadfamily 
\sphinxAtStartPar
协议名
&\sphinxstyletheadfamily 
\sphinxAtStartPar
参数
&\sphinxstyletheadfamily 
\sphinxAtStartPar
说明
\\
\hline
\sphinxAtStartPar
PPI\_P
&
\sphinxAtStartPar
1 bit Start + 1 bit Stop + 8 bit data + 1 bit even
&
\sphinxAtStartPar
PLC默认通讯协议
\\
\hline
\sphinxAtStartPar
other
&
\sphinxAtStartPar
1 bit Start + 1 bit Stop + 8 bit data
&
\sphinxAtStartPar
典型Modbus协议
\\
\hline
\end{tabular}
\par
\sphinxattableend\end{savenotes}

\begin{figure}[htbp]
\centering
\capstart

\noindent\sphinxincludegraphics[scale=0.5]{{11_扩展协议PPI}.jpg}
\caption{图 1.6.1 扩展协议PPI}\label{\detokenize{operation_guide:id15}}\end{figure}

\begin{figure}[htbp]
\centering
\capstart

\noindent\sphinxincludegraphics[scale=0.5]{{12_扩展协议other}.jpg}
\caption{图 1.6.2 扩展协议other}\label{\detokenize{operation_guide:id16}}\end{figure}

\sphinxAtStartPar
\sphinxstylestrong{6、密码修改设置}

\sphinxAtStartPar
密码修改界面,主要用于锁定参数设置菜单,在首次设置新密码或者二次修改密码时使用。

\begin{figure}[htbp]
\centering
\capstart

\noindent\sphinxincludegraphics[scale=0.5]{{14_密码修改界面1}.jpg}
\caption{图 1.7.1 密码修改界面}\label{\detokenize{operation_guide:id17}}\end{figure}

\begin{figure}[htbp]
\centering
\capstart

\noindent\sphinxincludegraphics[scale=0.5]{{15_密码输入界面2}.jpg}
\caption{图 1.7.2 新密码输入}\label{\detokenize{operation_guide:id18}}\end{figure}

\begin{figure}[htbp]
\centering
\capstart

\noindent\sphinxincludegraphics[scale=0.5]{{16_密码输入界面3}.jpg}
\caption{图 1.7.3 新密码确认}\label{\detokenize{operation_guide:id19}}\end{figure}

\begin{sphinxadmonition}{note}{注解:}\begin{itemize}
\item {} 
\sphinxAtStartPar
和输入密码界面操作一致,同样需要对输入的每位密码通过 \sphinxcode{\sphinxupquote{MENU}} 进行确认,输入完毕后按下 \sphinxcode{\sphinxupquote{ENTER}} 进行存储。

\end{itemize}
\end{sphinxadmonition}

\sphinxAtStartPar
\sphinxstylestrong{7、波特率设置}

\sphinxAtStartPar
配合 \sphinxstylestrong{4、工作模式} 来使用,详细配置请看以上章节4。

\begin{figure}[htbp]
\centering
\capstart

\noindent\sphinxincludegraphics[scale=0.5]{{17_波特率设置1}.jpg}
\caption{图 1.8.1 波特率设置1}\label{\detokenize{operation_guide:id20}}\end{figure}

\begin{figure}[htbp]
\centering
\capstart

\noindent\sphinxincludegraphics[scale=0.5]{{18_波特率设置2}.jpg}
\caption{图 1.8.2 波特率设置2}\label{\detokenize{operation_guide:id21}}\end{figure}

\begin{figure}[htbp]
\centering
\capstart

\noindent\sphinxincludegraphics[scale=0.5]{{19_波特率设置3}.jpg}
\caption{图 1.8.3 波特率设置3}\label{\detokenize{operation_guide:id22}}\end{figure}

\sphinxAtStartPar
\sphinxstylestrong{8、本机热点设置}

\sphinxAtStartPar
本机热点主要表现的是每台PLC的WIFI模块对外发射的 \sphinxstylestrong{AP\_ID}, 为每台PLC的唯一设备序列号标识。
通过Web端配网接口输入目标路由器账号和密码,就可以连接到云端与PLC进行数据交互。

\begin{figure}[htbp]
\centering
\capstart

\noindent\sphinxincludegraphics[scale=0.5]{{20_本机热点}.jpg}
\caption{图 1.9.1 本机热点设置}\label{\detokenize{operation_guide:id23}}\end{figure}

\begin{sphinxadmonition}{note}{注解:}\begin{itemize}
\item {} 
\sphinxAtStartPar
\sphinxcode{\sphinxupquote{热点}} 可以在配完网络后可以进行关闭,来降低设备功耗。

\end{itemize}
\end{sphinxadmonition}

\sphinxAtStartPar
\sphinxstylestrong{9、恢复出厂设置}

\sphinxAtStartPar
恢复出厂设置,主要用于PLC配置参数错乱、遗忘或者PLC部分功能异常时来使用,可能会解决您的大部分问题。
在当前界面连续2次按下 \sphinxcode{\sphinxupquote{ENTER}} 后将会触发功能。

\begin{figure}[htbp]
\centering
\capstart

\noindent\sphinxincludegraphics[scale=0.5]{{21_恢复出厂设置}.jpg}
\caption{图 1.10.1 恢复出厂设置}\label{\detokenize{operation_guide:id24}}\end{figure}

\begin{sphinxadmonition}{note}{注解:}\begin{itemize}
\item {} 
\sphinxAtStartPar
恢复出厂设置后,所有参数将会重置,包括但不限于:

\item {} 
\sphinxAtStartPar
①PLC启停状态。

\item {} 
\sphinxAtStartPar
②PLC通信方式。

\item {} 
\sphinxAtStartPar
③PLC工作模式。

\item {} 
\sphinxAtStartPar
④PLC扩展协议。

\item {} 
\sphinxAtStartPar
⑤PLC用户密码。

\item {} 
\sphinxAtStartPar
⑥PLC热点名称及用户自定义路由账号、密码。(云端账号和连接不会丢失)

\end{itemize}
\end{sphinxadmonition}

\sphinxAtStartPar
\sphinxstylestrong{10、无线网络配置步骤}
\begin{itemize}
\item {} 
\sphinxAtStartPar
首先打开需要连接无线网络的PLC,找到本机热点界面,查看本机的序列号标识。

\end{itemize}

\begin{figure}[htbp]
\centering
\capstart

\noindent\sphinxincludegraphics[scale=0.3]{{本机热点序列号识别}.jpg}
\caption{图 1.11.1 本机热点序列号标识}\label{\detokenize{operation_guide:id25}}\end{figure}
\begin{itemize}
\item {} 
\sphinxAtStartPar
其次用手机或有连接无线网功能的电脑连接对应本机热点,如下图 \sphinxstylestrong{PLC27\_AP}:

\end{itemize}

\begin{figure}[htbp]
\centering
\capstart

\noindent\sphinxincludegraphics[scale=0.5]{{连接WIFI}.jpg}
\caption{图 1.11.2 连接WIFI}\label{\detokenize{operation_guide:id26}}\end{figure}
\begin{itemize}
\item {} 
\sphinxAtStartPar
然后根据 \sphinxstylestrong{配网地址10.10.100.254/index\_cn.html} ,使用电脑浏览器或手机浏览器访问此地址,默认登录密码如下图所示,用户名和密码皆为 \sphinxstylestrong{admin}

\end{itemize}

\begin{figure}[htbp]
\centering
\capstart

\noindent\sphinxincludegraphics[scale=0.5]{{默认登录密码}.jpg}
\caption{图 1.11.3 默认登录密码}\label{\detokenize{operation_guide:id27}}\end{figure}
\begin{itemize}
\item {} 
\sphinxAtStartPar
最后进入配置界面后点击 \sphinxstylestrong{WIFI参数} (其它参数请勿修改,否则后果自负点击),找到 \sphinxstylestrong{STA} 参数设置 ,点击 \sphinxstylestrong{搜索} 找到需要连接的网络名称,输入密码后点击保存,至此完成该台PLC网络配置。

\end{itemize}

\begin{figure}[htbp]
\centering
\capstart

\noindent\sphinxincludegraphics[scale=0.5]{{配置界面}.jpg}
\caption{图 1.11.4 配置界面}\label{\detokenize{operation_guide:id28}}\end{figure}


\chapter{Indices and tables}
\label{\detokenize{index:indices-and-tables}}\begin{itemize}
\item {} 
\sphinxAtStartPar
\DUrole{xref,std,std-ref}{genindex}

\item {} 
\sphinxAtStartPar
\DUrole{xref,std,std-ref}{modindex}

\item {} 
\sphinxAtStartPar
\DUrole{xref,std,std-ref}{search}

\end{itemize}



\renewcommand{\indexname}{索引}
\printindex
\end{document}